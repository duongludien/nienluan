\documentclass[../thesis.tex]{subfiles}

\begin{document}

Kết quả đạt được của đề tài đáp ứng được mục tiêu đề ra ban đầu. Tuy nhiên, vẫn còn nhiều điểm cần khắc phục.

Về mặt dữ liệu, tập dữ liệu thu thập được chỉ giới hạn trong khu vực nội ô thành phố Cần Thơ, chưa thể tập hợp đầy đủ các loại biển báo quan trọng ở Việt Nam, số lượng dữ liệu mỗi loại biển báo cũng chưa thực sự đáp ứng được yêu cầu của việc phân lớp. Đa số các biển báo được chụp bằng điện thoại, góc nhìn thường là chính diện, chưa thể sử dụng được cho việc nhận dạng biển báo giao thông cho xe tự lái.

Mô hình nhận dạng đạt được mục tiêu đề ra ban đầu, đó là một mô hình nhận dạng vật thể theo một hướng tiếp cận mới, một giải thuật có hiệu quả cao. Cụ thể, giải thuật đạt được độ chính xác là 0.964 với thời gian chạy trung bình là 4 giây cho mỗi frame ảnh (không có GPU). Một số biển báo tương đồng nhau thì hiệu quả phân lớp chưa tốt.

Hướng phát triển trong tương lai:

\begin{itemize}
  \item Tiếp tục thu thập thêm dữ liệu về biển báo giao thông ở các khu vực khác, sử dụng camera hành trình để thu thập dữ liệu video thay vì chụp ảnh.
  \item Tiếp cận bài toán nhận dạng biển báo giao thông theo một hướng khác là giải thuật YOLO\cite{DBLP:journals/corr/RedmonDGF15}, hiện đang là một hiện tượng trong lĩnh vực Thị giác máy tính để giải quyết bài toán nhận dạng vật thể với thời gian thực.
\end{itemize}

\end{document}