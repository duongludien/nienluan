% oneside: 1 mặt
\documentclass[a4paper,12pt,oneside]{report}

% Tiếng Việt
\usepackage[utf8]{vietnam}

% Dùng để set margin
\usepackage[top=3cm,inner=3.5cm,outer=2cm,bottom=3cm,headheight=120pt]{geometry}		

\usepackage{graphicx}			% Dùng để chèn hình ảnh
\graphicspath{ {images/} }		% Folder chứa hình ảnh
\usepackage{subfiles}			% Dùng để chia thành các files nhỏ
\usepackage{fancyhdr}			% Dùng để tạo header và footer
\usepackage{emptypage}			% Không đánh số trang, header, footer với các trang trắng
\usepackage{tabularx}			% Dùng để chèn bảng

% Định dạng cho code
\usepackage{fancyvrb}			% Phiên bản nâng cao của verbatim, dùng để chèn code
\fvset{
	tabsize=4,					% Định tabsize 
	frame=single,				% Khung
}

% Hiển thị dòng kẻ để viết tay (phần Nhận xét của giảng viên)
\usepackage{pgffor, ifthen}
\newcommand{\notes}[3][\empty]{
	\noindent \vspace{15pt}\\
	\foreach \n in {1,...,#2}{
		\ifthenelse{
			\equal{#1}{\empty}
		}
		{\rule{#3}{0.5pt}\vspace{15pt}\\}
		{\rule{#3}{0.5pt}\vspace{15pt}\\}
	}
}

\title{Nhận dạng biển báo giao thông bằng Fast-RCNN}						% Tiêu đề luận văn
\author{Ngô Nguyễn Gia Bảo - B1401119 \and Dương Lữ Điện - B1401133}		% Người thực hiện 

% Header và footer 
\pagestyle{fancy}
\fancyhf{}

% Header và footer nếu in 1 mặt
\rhead{Giảng viên hướng dẫn:\\TS. Trần Công Án}						% Header bên phải
\lhead{Đề tài:\\Nhận dạng biển báo giao thông bằng Fast-RCNN}		% Header bên trái
\rfoot{\thepage}													% Footer bên phải
\lfoot{Ngô Nguyễn Gia Bảo - B1401119\\Dương Lữ Điện - B1401133}		% Footer bên trái

\begin{document}
\subfile{bia-ngoai}		% Gọi file bìa ngoài
\subfile{bia-trong}		% Gọi file bìa trong

\pagenumbering{roman}	% Kiểu số trang: i, ii, iii, iv, v,...

\chapter*{NHẬN XÉT CỦA GIẢNG VIÊN}
\subfile{chapters/nhan-xet-gv}

\chapter*{LỜI CẢM ƠN}
\subfile{chapters/loi-cam-on}

\tableofcontents
\listoffigures
\listoftables

\clearpage
\pagenumbering{arabic}	% Kiểu số trang: 1, 2, 3,...
\part{GIỚI THIỆU}
\subfile{chapters/gioi-thieu}

\part{NỘI DUNG}

\chapter{CƠ SỞ LÝ THUYẾT}
\subfile{chapters/chapter01}

\chapter{THIẾT KẾ MÔ HÌNH}
\subfile{chapters/chapter02}

\chapter{KIỂM THỬ VÀ ĐÁNH GIÁ}

\part{KẾT LUẬN}
\subfile{chapters/ket-luan}

\chapter*{THAM KHẢO}
\subfile{chapters/tham-khao}

\end{document}